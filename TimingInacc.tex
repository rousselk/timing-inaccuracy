\documentclass[a4paper,10pt]{article}

\usepackage[T1]{fontenc}

\title{Timing Inaccuracy in MSPSim Emulator: Consequences on COOJA Simulations}
\author{
K\'evin Roussel, Ye-Qiong Song and Olivier Zendra\\
LORIA/INRIA Nancy Grand-Est,\\
             Universit\'e de Lorraine,\\
             615, rue du Jardin Botanique,\\
             54600 Villers-L\`es-Nancy, France\\
\texttt{\{Kevin.Roussel,Ye-Qiong.Song,Olivier.Zendra\}@inria.fr}
}
\date{}

%%%%%%%%%%%%%%%%%%%%%%%%%%%%%%%%%%%%%%%%%%%%%%%%%%%%%%%%%%%%%%%%%%%%%%%%%%%%%%%%
%%%                               80 COLONNES                                %%%
%%%%%%%%%%%%%%%%%%%%%%%%%%%%%%%%%%%%%%%%%%%%%%%%%%%%%%%%%%%%%%%%%%%%%%%%%%%%%%%%

\begin{document}

\maketitle

\abstract{
Xxx
}

\section{COOJA and MSPSim}

Provided by the Contiki OS project~\cite{ContikiOS}, the COOJA network
simulator~\cite{Cooja} has become a widely used tool in the domain of
Wireless Sensor Networks (WSN). The research community especially uses
it extensively to perform simulations of small to relatively large wireless
networks of connected devices embedding sensors and/or actuators, commonly
named \emph{``motes''}, and thus to develop, debug and evaluate projects
based in the WSN technology.

COOJA itself is a Java-based application, providing three main features:
\begin{enumerate}
\item a graphical user interface (GUI, based on Java's standard Swing toolkit)
to design, run, and analyze WSN simulations;
\item simulation of the radio medium underlying the wireless communications
of sensor networks;
\item an extensible framework, which allows integration of external tools
to provide additional features to the COOJA application.
\end{enumerate}
This last feature is used to allow COOJA to actually emulate the various
motes constituting the WSNs. It indeed embeds and uses dedicated emulator
programs that perform cycle-exact emulation of the chips with which motes
are built: microcontroller units (MCUs), radio transceivers, etc.

This emulation mechanism is one of the main strong points of COOJA:
thanks to it, very fine-grained, precise and low-level simulations
can be performed; this is why COOJA has become a tool of choice especially
for debugging and evaluating WSN-related software (which happens to be
often based on Contiki OS).

Current versions of COOJA make use of two different emulator software
packages: Avrora for emulation of Atmel AVR-based devices, and
MSPSim~\cite{MSPSim} for emulation of TI MSP430-based devices.

Of these two emulators, MSPSim is currently the most used in literature
including COOJA-based simulations, since motes based on MSP430 MCUs are
more commonly distributed and used: we can especially think to the pervasive
TelosB/SkyMote family, or to Zolertia's Z1 platform.

\section{Timing Inaccuracy Problem in MSPSim}



\section{Consequences}




%%%%%%%%%%%%%%%%%%%%%%%%%%%%%%%%%%%%%%%%%%%%%%%%%%%%%%%%%%%%%%%%%%%%%%%%%%%%%


\vfill
\bibliographystyle{unsrt}
{\small
\bibliography{TimingInacc}}

\end{document}
